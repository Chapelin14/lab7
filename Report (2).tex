\documentclass{article}
\usepackage[utf8]{inputenc}
\usepackage[english,russian]{babel}

\usepackage{indentfirst}
\usepackage{misccorr}
\usepackage{graphicx}
\graphicspath{}
\DeclareGraphicsExtensions{.pdf,.png,.jpg}

\usepackage{amsmath}
\usepackage{minted}
\begin{document}
\huge
\textbf{Лабораторная работа № 7}
\Large

Выполнил студент группы 428

Комарова Анна Александровна

\LARGE
\textbf{Вариант № 04}

\Large
Решить с помощью чисто неявной схемы задачу Коши \( y'''-2y''+y'-2y=cos(x)\cdot exp(2x),  y(0)=1,{y}'(0)=0,{y}''(0)=0,x\in [0,2]\)
c заданной относительной точностью 0,01.

\LARGE
\textbf{Теоретическая часть}

\Large
%\textbf{Разностные схемы. Чисто неявная схема}
Любое дифференциальное уравнение  m-го порядка \(y^{(m)}=f(x,y,y',...,y^{(m-1)})\) можно свести к системе, состоящей из m уравнений 1-го порядка при помощи замен. 
\(\\y_{1}=y',\\
y_{2}=y''=y_{1}'\\
y_{3}=y'''=y_{2}',\\
...\\
y_{m}=y^{(m)}=y_{m-1}'.\\
\)
В результате получаем систему:

\(\begin{cases}
 y'=y_{1}, \\ 
 y_{1}'=y_{2},\\ 
 y_{2}'=y_{3}', \\ 
 ... \\ 
 y_{m-1}'=f(x,y,...,y_{m-1}). 
\end{cases} \)

Решением системы и дифференциального уравнения является m функций 
\(y, y_{1}=y',...,y_{m}=y_{m-1}'\). Множество точек \(y(x),y'(x),...,y_{m-1}'(x),x\in [a,b]\) образуют кривую в пространстве \(R^m\), которую называют фазовой траекторией.

\textbf{Задача Коши} состоит в нахождении решения дифференциального уравнения, удовлетворяющего начальным условиям. При численном решении задачи ищется последовательность векторов (приближений для значений решения) на множестве точек сетки \(x_{i}:x_{i+1}=x_{i}+h_{i}, i=\overline{0,N-1}. \)

Для задачи 
\( \begin{cases}
 \frac{\partial \vec{u}}{\partial x}+A(x)\vec{u}=\vec{\varphi} (x),& x>0 \\ 
 \vec{u}(0)=\vec{u}_{0}. 
\end{cases} \) чисто неявная разностная схема - 
\(\begin{cases}
 \frac{\vec{y}_{n+1}-\vec{y}_{n}}{h}+A\vec{y}_{n}=\vec{\varphi}_{n}\\ 
n=0,1,2,..., \vec{y}_{0}=\vec{u}_{0} 
\end{cases} \), где A - матрица порядка m\times m, \(\vec{y}_{n},\vec{\varphi}_{n}\) - векторы размерности m.

Для оценки погрешности применяется правило Рунге \(\frac{|y_{i,h}-y_{i,h/2}|}{2^{p-1}}\), p - порядок точности численного метода.

\LARGE
\textbf{Практическая часть}

\Large
Для применения разностной схемы дифференциальное уранвение 3-го порядка \( y'''-2y''+y'-2y=cos(x)\cdot exp(2x)\) было сведено к системе ДУ 1-го порядка: 

\(\begin{cases}
 y'=y_{1}, \\ 
 y_{1}'=y_{2},\\ 
 y_{2}'=2y_{2}-y_{1}+2y+cos(x)\cdot exp(2x),\\
 y(0)=1,y_{1}(0)=0,y_{2}(0)=0.
\end{cases}\)

\(A=\begin{pmatrix}
0 & -1 & 0 \\ 
0 & 0 & -1\\ 
-2 & 1 & -2
\end{pmatrix}\)

Отрезок \(x\in [0,2]\) разбивается на множество точек с начальным шагом h.
В цикле for функции \textit{main()} вычисляются приближенные значения решений системы в каждой точке сетки с шагом h и с шагом h/2, затем вычисляется относительная погрешность по правилу Рунге ( \(\underset{i}{max}|y_{i,h}-y_{i,h/2}|=\varepsilon\)). Для вычисления значений \(\vec{y}_{n+1}\) применяется схема \textit{предиктор-корректор}. Первый шаг осуществляется методом Эйлера \(y_{n+1}=y_{n}+hf(x_{n},y_{n})\) (вычисляется грубое приближение \(\vec{y}_{n+1}\), которое записывается в y1): 
\begin{minted}{c++}
void Euler_method(double** y, double* y1, int k){
	y1[0]=y[0][k-1]+h*y[1][k-1];
	y1[1]=y[1][k-1]+h*y[2][k-1];
	y1[2]=y[2][k-1]+h*(2*y[2][k-1]-y[1][k-1]+
	2*y[0][k-1]+cos(a+h*(k-1))*exp(2*(a+h*(k-1))));
}
\end{minted}
На втором шаге предсказание уточняется и выполняется чисто неявная схема (в качестве аргумента y1 принимаются предсказанные на первом шаге значения):
\begin{minted}{c++}
void implicit_scheme(double** y, double* y1, int k){
	y[0][k]=h*0+y[0][k-1]-h*(A[0][0]*y1[0]+
	        A[0][1]*y1[1]+A[0][2]*y1[2]);
	y[1][k]=h*0+y[1][k-1]-h*(A[1][0]*y1[0]+
	        A[1][1]*y1[1]+A[1][2]*y1[2]);
	y[2][k]=h*(cos(a+h*(k-1))*exp(2*(a+h*(k-1))))+
	        y[2][k-1]-h*(A[2][0]*y1[0]+A[2][1]*y1[1]+
	        A[2][2]*y1[2]);
}
\end{minted}
Оба шага объеденены в методе \textit{void Y(double** y, double h)}, который в зависимости от шага сетки вычисляет приближенное решение.

\LARGE
\textbf{Результаты}

\Large
В результате работы программы была решена задача Коши для дифференциального уравнения \( y'''-2y''+y'-2y=cos(x)\cdot exp(2x)\) с относительной точностью 0,01. Ниже приведены графики решения \(y(x),y'(x)\), а также фазовая траектория.

\includegraphics[scale=0.9]{1}

\includegraphics[scale=0.9]{2}
\end{document}
