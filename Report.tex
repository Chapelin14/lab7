\documentclass{article}
\usepackage[utf8]{inputenc}
\usepackage[english,russian]{babel}

\usepackage{indentfirst}
\usepackage{misccorr}
\usepackage{graphicx}
\usepackage{amsmath}
\begin{document}
\huge
\textbf{Лабораторная работа № 7}
\Large

Выполнил студент группы 428

Мунин Сергей Александрович

\LARGE
\textbf{Вариант № 16}

\Large
Решить с помощью чисто неявной схемы задачу Коши \(4y′′+ 5y′+ 2y=x·exp(−x),  y(0)=1,{y}'(0)=0,x\in [0,2]\)
c заданной относительной точностью 0,01.

\LARGE
\textbf{Теоретическая часть}

\Large
%\textbf{Разностные схемы. Чисто неявная схема}
Любое дифференциальное уравнение  m-го порядка \(y^{(m)}=f(x,y,y',...,y^{(m-1)})\) можно свести к системе, состоящей из m уравнений 1-го порядка при помощи замен. 
\(\\y_{1}=y',\\
y_{2}=y''=y_{1}'\\
y_{3}=y'''=y_{2}',\\
...\\
y_{m}=y^{(m)}=y_{m-1}'.\\
\)
В результате получаем систему:
\(\\
\begin{cases}
 y'=y_{1}, \\ 
 y_{1}'=y_{2},\\ 
 y_{2}'=y_{3}', \\ 
 ... \\ 
 y_{m-1}'=f(x,y,...,y_{m-1}). 
\end{cases} \\ \)
Решением системы и дифференциального уравнения является m функций 
\(y, y_{1}=y',...,y_{m}=y_{m-1}'\).

\textbf{Задача Коши} состоит в нахождении решения дифференциального уравнения, удовлетворяющего начальным условиям. При численном решении задачи ищется последовательность векторов (приближений для значений решения) на множестве точек сетки \(x_{i}:x_{i+1}=x_{i}+h_{i}, i=\overline{0,N-1}. \)

Для задачи 
\( \begin{cases}
 \frac{\partial \vec{u}}{\partial x}+A(x)\vec{u}=\vec{\varphi} (x),& x>0 \\ 
 \vec{u}(0)=\vec{u}_{0}. 
\end{cases} \) чисто неявная разностная схема 
\(\begin{cases}
 \frac{\vec{y}_{n+1}-\vec{y}_{n}}{h}+A\vec{y}_{n}=\vec{\varphi}_{n}\\ 
n=0,1,2,..., \vec{y}_{0}=\vec{u}_{0} 
\end{cases} \), где A - матрица порядка m\times m, \(\vec{y}_{n},\vec{\varphi}_{n}\) - векторы размерности m.

\LARGE
\textbf{Практическая часть}
\Large

Программа состоит из двух функций:
1.main
2.rightpart

В функции main находится неявный метод. В ней задаются две квадратичные матрицы для записи предыдущего и последующего состояний.
 Далее с помощью цикла находится значение функции при заданном х. Сама переменная х изменяется с каждым проходом по циклу на величину шага h.
 Для выполнения условия остановки цикла находим норму матриц.
 После этого переписываем значения матриц в другие матрицы для надежности и выводим их.
 
В функции rightpat происходит вычисление y неявного метода. Для этого на вход подается значение x, а так же значения первой и второй производных.
 Из уравнения извлекается переменная y и вычисляется с помощью заданных пареметров.
 
\LARGE
\textbf{Результаты}
\Large

В результате работы программы с помощью чисто неявной схемы была решена задача Коши, а именно, решено дифференциальное уравнение.
Значение у на заданном интервале x\in [0,2]\ равно  y = -0.0019995.

\end{document}
